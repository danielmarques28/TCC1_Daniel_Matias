\chapter{Considerações Finais}

Para alcançar os objetivos descritos na seção \ref{section:objetivos}, que referem-se ao desenvolvimento de um sistema de recomendação, foi realizada inicialmente toda pesquisa bibliográfica necessária, além da realização da documentação referente a arquitetura do projeto, e dessa forma a descrição de como seria todo funcionamento do sistema e seu processo de análise, projeto e desenvolvimento com uma metodologia adequada.

O objetivo principal desse trabalho é a construção de um sistema de recomendação robusto que funcionará como um serviço mais interessante para o ambiente virtual da Liva.vc, em que possa proporcionar para o usuário uma melhor experiência de uso com recomendações mais coerente com as demandas e preferências de seus usuários.

Com a realização de implantação de um protótipo implementado para fins de simulação em ambiente da Liva, foi possível, com a análise dos dados coletados durante o experimento, a confirmação da viabilidade parcial de realização do projeto. Como o período do experimento foi pequeno, não foi possível apresentar um espaço amostral maior em que o processo de escolha de um imóvel para compra por um usuário é longo e necessita de muitas interações, o que normalmente leva um tempo maior.

Para a segunda parte da monografia espera-se, com o desenvolvimento e aplicação do sistema de recomendação em ambiente de produção sítio virtual da Liva, a apresentação de um melhor resultado obtido pela mesma métrica adotada por Li et al. (\citeyear{Summo:2017}). Com ela será possível mensurar melhor a eficiência de um recomendador comparado ao outro previamente atribuído.

Dessa forma, ao se apresentar melhores resultados iniciais é possível manter a motivação e a expectativa de que o desenvolvimento e implantação do novo sistema de recomendação no modelo híbrido proposto obterá melhores resultados para os usuários do ambiente virtual da Liva.


% Como dito anteriormente no primeiro capítulo, o objetivo principal desse trabalho é a construção de um sistema de recomendação robusto que funcionará como um serviço para a plataforma Liva.vc, em que possa proporcionar para o usuário uma melhor experiencia na plataforma. Assim, primeiramente, para um melhor entendimento sobre o assunto, foi realizada toda pesquisa bibliográfica necessárias para sua implementação, além de toda documentação referente a arquitetura e dessa forma a descrição de todo funcionamento do sistema de recomendação a ser desenvolvido.

% Com a realização de implantação de um protótipo implementado para fins de simulação em ambiente da Liva, foi possível, com a analise dos dados coletados durante o experimento, a confirmação de viabilidade parcial de realização do projeto. Como o período do experimento foi pequeno, não foi possível apresentar uma bom espaço amostral dado que o processo de escolha de um imóvel para compra por um usuário é longo e necessita de muitas interações o que normalmente leva um maior período de tempo.

% Para a segunda parte da monografia espera-se, com o desenvolvimento e aplicação do sistema de recomendação em ambiente de produção na Liva, a apresentação de um melhor resultado obtido pela mesma métrica adotada por Li et al. \citeyear{Summo:2017}, explicada na seção \ref{proposta}. Com ela é possível medir a eficiência de um recomendador comparado ao outro previamente atribuído. Dessa forma ao se apresentar melhores resultados, será possível concluir que a utilização de abordagens para sistemas de recomendação seguindo um modelo hibrido é melhor ao contexto da Liva.

% Na seção a seguir será apresentada uma simulação com o objetivo de validar a viabilidade da proposta. Foi realizada a criação de um protótipo, implantado no site da liva durante um período de tempo. Ele serviu para apresentação dos resultados parciais, possibilitando sua analise e assim uma conclusão informando se atendeu ou não ao esperado.
