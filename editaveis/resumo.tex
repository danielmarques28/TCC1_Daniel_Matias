\begin{resumo}

Diante da grande quantidade e variedade de informações na Internet atualmente, criou-se a necessidade da existência de sistemas que ajudem os usuários a encontrarem o que eles desejam de acordo com sua preferência, para assim diminuir o tempo gasto em pesquisas nesse ambiente. Em sítios virtuais de comércio eletrônico, por exemplo, há centenas ou milhares de produtos à mostra para os usuários, e isso dificulta o trabalho de encontrar o produto adequado as particulares de cada cliente. O presente trabalho trata-se do desenvolvimento de um sistema de recomendação de imóveis, no contexto de negócios do tipo B2B (\textit{Business to business}). Por meio de um sítio virtual de \textit{e-commerce} de venda de imóveis para imobiliárias apresentarem suas propriedades e obterem visibilidade de seus possíveis clientes, utilizando de um sistema de recomendação que procuraria atendê-los de maneira mais eficiente e diminuindo o esforço de procura ao que atenderia seus interesses particulares para aquisição de um novo imóvel. O objetivo principal seria auxiliar os clientes que visitassem este sítio virtual em localizar o imóvel desejado de forma mais eficiente, desenvolvendo um sistema de recomendação mais robusto, comparado aos sistemas já existente. As recomendações de imóveis terão como base as características desses clientes adquiridas pelo sistema recomendador. Ele seguirá uma abordagem híbrida, ou seja, contemplando duas ou mais técnicas de pesquisas, gerando melhores resultados aos clientes usuários desse sistema. Foi realizado um estudo inicial por meio de pesquisas bibliográficas sobre conceitos, abordagens e técnicas utilizadas em sistemas de recomendação no contexto imobiliário para geração de uma nova proposta de sistema de recomendação imobiliário que estará sendo desenvolvido neste trabalho.

 \vspace{\onelineskip}
    
 \noindent
 \textbf{Palavras-chave}: aprendizado de máquina. sistema de recomendação. comércio eletrônico. imobiliária.
\end{resumo}
